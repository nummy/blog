\documentclass[]{article}
\usepackage{lmodern}
\usepackage{amssymb,amsmath}
\usepackage{ifxetex,ifluatex}
\usepackage{fixltx2e} % provides \textsubscript
\ifnum 0\ifxetex 1\fi\ifluatex 1\fi=0 % if pdftex
  \usepackage[T1]{fontenc}
  \usepackage[utf8]{inputenc}
\else % if luatex or xelatex
  \ifxetex
    \usepackage{mathspec}
  \else
    \usepackage{fontspec}
  \fi
  \defaultfontfeatures{Ligatures=TeX,Scale=MatchLowercase}
\fi
% use upquote if available, for straight quotes in verbatim environments
\IfFileExists{upquote.sty}{\usepackage{upquote}}{}
% use microtype if available
\IfFileExists{microtype.sty}{%
\usepackage[]{microtype}
\UseMicrotypeSet[protrusion]{basicmath} % disable protrusion for tt fonts
}{}
\PassOptionsToPackage{hyphens}{url} % url is loaded by hyperref
\usepackage[unicode=true]{hyperref}
\hypersetup{
            pdfborder={0 0 0},
            breaklinks=true}
\urlstyle{same}  % don't use monospace font for urls
\usepackage{color}
\usepackage{fancyvrb}
\newcommand{\VerbBar}{|}
\newcommand{\VERB}{\Verb[commandchars=\\\{\}]}
\DefineVerbatimEnvironment{Highlighting}{Verbatim}{commandchars=\\\{\}}
% Add ',fontsize=\small' for more characters per line
\newenvironment{Shaded}{}{}
\newcommand{\KeywordTok}[1]{\textcolor[rgb]{0.00,0.44,0.13}{\textbf{#1}}}
\newcommand{\DataTypeTok}[1]{\textcolor[rgb]{0.56,0.13,0.00}{#1}}
\newcommand{\DecValTok}[1]{\textcolor[rgb]{0.25,0.63,0.44}{#1}}
\newcommand{\BaseNTok}[1]{\textcolor[rgb]{0.25,0.63,0.44}{#1}}
\newcommand{\FloatTok}[1]{\textcolor[rgb]{0.25,0.63,0.44}{#1}}
\newcommand{\ConstantTok}[1]{\textcolor[rgb]{0.53,0.00,0.00}{#1}}
\newcommand{\CharTok}[1]{\textcolor[rgb]{0.25,0.44,0.63}{#1}}
\newcommand{\SpecialCharTok}[1]{\textcolor[rgb]{0.25,0.44,0.63}{#1}}
\newcommand{\StringTok}[1]{\textcolor[rgb]{0.25,0.44,0.63}{#1}}
\newcommand{\VerbatimStringTok}[1]{\textcolor[rgb]{0.25,0.44,0.63}{#1}}
\newcommand{\SpecialStringTok}[1]{\textcolor[rgb]{0.73,0.40,0.53}{#1}}
\newcommand{\ImportTok}[1]{#1}
\newcommand{\CommentTok}[1]{\textcolor[rgb]{0.38,0.63,0.69}{\textit{#1}}}
\newcommand{\DocumentationTok}[1]{\textcolor[rgb]{0.73,0.13,0.13}{\textit{#1}}}
\newcommand{\AnnotationTok}[1]{\textcolor[rgb]{0.38,0.63,0.69}{\textbf{\textit{#1}}}}
\newcommand{\CommentVarTok}[1]{\textcolor[rgb]{0.38,0.63,0.69}{\textbf{\textit{#1}}}}
\newcommand{\OtherTok}[1]{\textcolor[rgb]{0.00,0.44,0.13}{#1}}
\newcommand{\FunctionTok}[1]{\textcolor[rgb]{0.02,0.16,0.49}{#1}}
\newcommand{\VariableTok}[1]{\textcolor[rgb]{0.10,0.09,0.49}{#1}}
\newcommand{\ControlFlowTok}[1]{\textcolor[rgb]{0.00,0.44,0.13}{\textbf{#1}}}
\newcommand{\OperatorTok}[1]{\textcolor[rgb]{0.40,0.40,0.40}{#1}}
\newcommand{\BuiltInTok}[1]{#1}
\newcommand{\ExtensionTok}[1]{#1}
\newcommand{\PreprocessorTok}[1]{\textcolor[rgb]{0.74,0.48,0.00}{#1}}
\newcommand{\AttributeTok}[1]{\textcolor[rgb]{0.49,0.56,0.16}{#1}}
\newcommand{\RegionMarkerTok}[1]{#1}
\newcommand{\InformationTok}[1]{\textcolor[rgb]{0.38,0.63,0.69}{\textbf{\textit{#1}}}}
\newcommand{\WarningTok}[1]{\textcolor[rgb]{0.38,0.63,0.69}{\textbf{\textit{#1}}}}
\newcommand{\AlertTok}[1]{\textcolor[rgb]{1.00,0.00,0.00}{\textbf{#1}}}
\newcommand{\ErrorTok}[1]{\textcolor[rgb]{1.00,0.00,0.00}{\textbf{#1}}}
\newcommand{\NormalTok}[1]{#1}
\IfFileExists{parskip.sty}{%
\usepackage{parskip}
}{% else
\setlength{\parindent}{0pt}
\setlength{\parskip}{6pt plus 2pt minus 1pt}
}
\setlength{\emergencystretch}{3em}  % prevent overfull lines
\providecommand{\tightlist}{%
  \setlength{\itemsep}{0pt}\setlength{\parskip}{0pt}}
\setcounter{secnumdepth}{0}
% Redefines (sub)paragraphs to behave more like sections
\ifx\paragraph\undefined\else
\let\oldparagraph\paragraph
\renewcommand{\paragraph}[1]{\oldparagraph{#1}\mbox{}}
\fi
\ifx\subparagraph\undefined\else
\let\oldsubparagraph\subparagraph
\renewcommand{\subparagraph}[1]{\oldsubparagraph{#1}\mbox{}}
\fi

% set default figure placement to htbp
\makeatletter
\def\fps@figure{htbp}
\makeatother


\date{}

\begin{document}

通常情况下,Django提供的User模型能够满足我们大部分的需求,但是有时候我们需要给User添加一些格外的功能和信息。

Django支持两种方式来扩展User模型。

\begin{itemize}
\item
  \textbf{代理模型}
  如果只是需要添加一些功能性操作,可以基于User创建一个代理模型。
\item
  \textbf{关联模型}
  如果是需要添加格外的字段信息,则可以创建一个与User关联的模型,两者之间的关系为1对1。这个模型通常称之为profile
  model,因为大部分情况下这些格外信息都与认证无关。
\end{itemize}

\subsubsection{关联模型}\label{header-n15}

下面创建一个关联模型,给User添加手机信息。

\begin{Shaded}
\begin{Highlighting}[]
\KeywordTok{class}\NormalTok{ Profile(models.Model):}
\NormalTok{    user }\OperatorTok{=}\NormalTok{ models.OneToOneField(User)}
\NormalTok{    mobile }\OperatorTok{=}\NormalTok{ models.CharField(max_length}\OperatorTok{=}\DecValTok{15}\NormalTok{, verbose_name}\OperatorTok{=}\StringTok{u'手机号'}\NormalTok{)}

    \KeywordTok{class}\NormalTok{ Meta:}
\NormalTok{        verbose_name }\OperatorTok{=} \StringTok{u'附加信息'}
\NormalTok{        verbose_name_plural }\OperatorTok{=} \StringTok{u'附加信息'}
\end{Highlighting}
\end{Shaded}

假设数据库中已经有一个用户同时具有User和Profile模型,则可以Django的关联模型获取数据:

\begin{Shaded}
\begin{Highlighting}[]
\OperatorTok{>>>}\NormalTok{ u }\OperatorTok{=}\NormalTok{ User.objects.get(username}\OperatorTok{=}\StringTok{'fsmith'}\NormalTok{)}
\OperatorTok{>>>}\NormalTok{ mobile }\OperatorTok{=}\NormalTok{ u.profile.mobile }
\end{Highlighting}
\end{Shaded}

为了在管理后台中将Profile作为一个字段添加到User管理页面中,需要定义一个
\texttt{InlineModelAdmin}, 并将它添加到UserAdmin类中。

\begin{Shaded}
\begin{Highlighting}[]
\ImportTok{from}\NormalTok{ django.contrib }\ImportTok{import}\NormalTok{ admin}
\ImportTok{from}\NormalTok{ django.contrib.auth.admin }\ImportTok{import}\NormalTok{ UserAdmin}
\ImportTok{from}\NormalTok{ django.contrib.auth.models }\ImportTok{import}\NormalTok{ User}
\ImportTok{from}\NormalTok{ .models }\ImportTok{import}\NormalTok{ Profile}

\KeywordTok{class}\NormalTok{ ProfileInline(admin.StackedInline):}
\NormalTok{    model }\OperatorTok{=}\NormalTok{ Profile}
\NormalTok{    can_delete }\OperatorTok{=} \VariableTok{False}
\NormalTok{    verbose_name_plural }\OperatorTok{=} \StringTok{'附加信息'}

\CommentTok{# Define a new User admin}
\KeywordTok{class}\NormalTok{ UserAdmin(UserAdmin):}
\NormalTok{    inlines }\OperatorTok{=}\NormalTok{ (ProfileInline,)}

\CommentTok{# Re-register UserAdmin}
\NormalTok{admin.site.unregister(User)}
\NormalTok{admin.site.register(User, UserAdmin)}
\end{Highlighting}
\end{Shaded}

这些profile模型并没有什么特别之处,它们只是与User模型存在一对一的关联而已。因此,当创建用户的时候,profile并不会自动创建,可以通过\texttt{django.db.models.signals.post\_save}来创建或者更新profile模型。

\begin{Shaded}
\begin{Highlighting}[]
\CommentTok{# coding:utf-8}
\ImportTok{from}\NormalTok{ __future__ }\ImportTok{import}\NormalTok{ unicode_literals}
\ImportTok{from}\NormalTok{ django.db }\ImportTok{import}\NormalTok{ models}
\ImportTok{from}\NormalTok{ django.contrib.auth.models }\ImportTok{import}\NormalTok{ User}
\ImportTok{from}\NormalTok{ django.db.models.signals }\ImportTok{import}\NormalTok{ post_save}


\CommentTok{# Create your models here.}
\KeywordTok{class}\NormalTok{ Profile(models.Model):}
\NormalTok{    user }\OperatorTok{=}\NormalTok{ models.OneToOneField(User)}
\NormalTok{    mobile }\OperatorTok{=}\NormalTok{ models.CharField(max_length}\OperatorTok{=}\DecValTok{15}\NormalTok{, verbose_name}\OperatorTok{=}\StringTok{u'手机号'}\NormalTok{)}

    \KeywordTok{class}\NormalTok{ Meta:}
\NormalTok{        verbose_name }\OperatorTok{=} \StringTok{u'附加信息'}
\NormalTok{        verbose_name_plural }\OperatorTok{=} \StringTok{u'附加信息'}

    \KeywordTok{def}\NormalTok{ save(}\VariableTok{self}\NormalTok{, }\OperatorTok{*}\NormalTok{args, }\OperatorTok{**}\NormalTok{kwargs):}
        \ControlFlowTok{if} \KeywordTok{not} \VariableTok{self}\NormalTok{.pk:}
            \ControlFlowTok{try}\NormalTok{:}
\NormalTok{                p }\OperatorTok{=}\NormalTok{ Profile.objects.get(user}\OperatorTok{=}\VariableTok{self}\NormalTok{.user)}
                \VariableTok{self}\NormalTok{.pk }\OperatorTok{=}\NormalTok{ p.pk}
            \ControlFlowTok{except}\NormalTok{ Profile.DoesNotExist:}
                \ControlFlowTok{pass}
        \BuiltInTok{super}\NormalTok{(Profile, }\VariableTok{self}\NormalTok{).save(}\OperatorTok{*}\NormalTok{args, }\OperatorTok{**}\NormalTok{kwargs)}


\KeywordTok{def}\NormalTok{ create_user_profile(sender, instance, created, }\OperatorTok{**}\NormalTok{kwargs):}
    \BuiltInTok{print}\NormalTok{(kwargs)}
    \ControlFlowTok{if}\NormalTok{ created:}
\NormalTok{        profile, created }\OperatorTok{=}\NormalTok{ Profile.objects.get_or_create(user}\OperatorTok{=}\NormalTok{instance)}


\NormalTok{post_save.}\ExtensionTok{connect}\NormalTok{(create_user_profile, sender}\OperatorTok{=}\NormalTok{User)}
\end{Highlighting}
\end{Shaded}

\end{document}
